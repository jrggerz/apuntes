\documentclass{article}
\usepackage[utf8]{inputenc}
\usepackage[spanish]{babel}
\usepackage{blindtext}
\usepackage{outlines}
\usepackage{authblk}
\usepackage{multicol}
\usepackage{geometry}
\usepackage{amssymb}
\usepackage{amsmath}
\usepackage{tabto}
\usepackage{biblatex} 
\addbibresource{ed.bib} %Import the bibliography file
\geometry{
 letterpaper,
 left=25mm,
 top=25mm,
 bottom=25mm,
 right=30mm
 }
\title{Funciones linealmente independientes y wronskiano.}
\author{Jorge Ruiz López}
\affil{Facultad de Ingeniería UNAM}
\date{Abril 2022}
\begin{document}
\maketitle
\section{Dependencia e independencia lineal:}
\begin{description}
\item[Definición 1.1:]Se dice que un conjunto de funciones $f_1(x), f_2(x),..., f_n(x)$ es \textbf{linealmente dependiente} en un intervalo $I$ si existen constantes $c_1, c_2,...,c_n$ no todas cero, tales que
\begin{center}
    $c_1f_1(x)+ c_2f_2(x)+...+c_nf_n(x) = 0$
\end{center}
para toda $x$ en el intervalo. Si el conjunto de funciones no es linealmente dependiente en el intervalo, se dice que es linealmente independiente.
\end{description}
\section{Wronskiano:}
\begin{description}
\item[Definición 1.2:]Suponga que cada una de las funciones $f_1(x), f_2(x), . . . , f_n(x)$ tiene al menos $n-1$ derivadas. El determinante
\begin{center}
\begin{equation}W(f_1, f_2, ...,f_n)= \begin{vmatrix}
f_1 & f_2 & ... & f_n\\
f_1' & f_2' & ... & f_n'\\
\vdots & \vdots & & \vdots\\
f_1^{n-1} & f_2^{n-1} & ... & f_n^{n-1} 
\end{vmatrix}
\end{equation}
\end{center}
\item[Criterio para soluciones linealmente independientes]Sean $y_l, y_2, . . . , y_n$ $n$ soluciones de la ecuación diferencial lineal homogénea de n-ésimo orden en el intervalo $I$. El conjunto de soluciones es linealmente independiente en $I$ si y sólo si $W(y_l, y_2, . . . ,y_n)\neq 0$   para toda $x$ en el intervalo.
\end{description}
\end{document}
