\documentclass{article}
\usepackage[utf8]{inputenc}
\usepackage[spanish]{babel}
\usepackage{blindtext}
\usepackage{outlines}
\usepackage{authblk}
\usepackage{multicol}
\usepackage{geometry}
\usepackage{amssymb}
\usepackage{tabto}
\usepackage{biblatex} 
\addbibresource{ed.bib} %Import the bibliography file
\geometry{
 letterpaper,
 left=25mm,
 top=25mm,
 bottom=25mm,
 right=30mm
 }
\title{Problema de valor inicial.}
\author{Jorge Ruiz López}
\affil{Facultad de Ingeniería UNAM}
\date{Marzo 2022}
\begin{document}
\maketitle
\section{Problema de valor inicial.}
\begin{description}
\item[Definición 1.1:]\cite{carmona} \emph{Problema con valor inicial} es la ecuación diferencial acompañada de condiciones iniciales.\\
\item[Ejemplo]
Resolver la ecuación diferencial:
\begin{center}$y'-4xy = 0$\end{center}
Para la condición inicial: $y = \frac{1}{5}$ cuando $x = 0$, o bien, brevemente:
\begin{center}$y(0)=\frac{1}{5} $\end{center}
La ecuación puede escribirse como:
\begin{center}$dy=4xy\cdot dx$,  o, $\frac{dy}{y} = 4x\cdot dx$\end{center}
Integrando ambos lados de la igualdad tenemos:\\
\begin{center}$ln(y) = 2x^2+c$\\$y=ce^{2x^{2}}$ \end{center}
Sustituyendo los valores del punto $(0, \frac{1}{5})$, tenemos que: $\frac{1}{5} = ce^0 \rightarrow c=\frac{1}{5}$\\
Entonces la solución particular es:
\begin{center}$y=\frac{1}{5}e^{2x^2}$\end{center}
\item[Ejemplo 2:]
Resolver la siguiente ecuación diferencial:
\begin{center}$y''= x$, para $y(-2) = 4$\\, $y'(0)=1$\end{center}
Integrando ambos lados de la ecuación tenemos:
\begin{center}$y'=\frac{x^2}{2}+c_1$\end{center}
Volviendo a integrar:
\begin{center}$y=\frac{x^3}{6}+c_1x+c_2$, es solución general\end{center}
Aplicando las condiciones iniciales dadas:
\begin{center}para $y':$ \tab $1=0+c_1 \rightarrow c_1 =1$\end{center}
\begin{center}para $y:$ \tab $4=\frac{-8}{6}-2(1)+c_2$\\$c_2 = \frac{22}{3}$\\$\blacksquare y= \frac{x^3}{6}+x+\frac{22}{3}$ es solución particular\end{center}
Comprobación: derivando la solución particular y sustituyéndola en la ecuación, debe satisfacerla:
\begin{center}$y'= \frac{x^2}{2}+ 1$\\$y''=x$\end{center}

\end{description}
\section{Observación:}Se necesita igual número de condiciones iniciales que el del orden de la ecuación diferencial.
\printbibliography
\end{document}
