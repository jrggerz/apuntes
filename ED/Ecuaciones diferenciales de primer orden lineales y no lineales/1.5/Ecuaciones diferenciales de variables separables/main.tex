\documentclass{article}
\usepackage[utf8]{inputenc}
\usepackage[spanish]{babel}
\usepackage{blindtext}
\usepackage{outlines}
\usepackage{authblk}
\usepackage{multicol}
\usepackage{geometry}
\usepackage{amssymb}
\usepackage{amsmath}
\usepackage{tabto}
\usepackage{biblatex} 
\addbibresource{ed.bib} %Import the bibliography file
\geometry{
 letterpaper,
 left=25mm,
 top=25mm,
 bottom=25mm,
 right=30mm
 }
\title{Ecuaciones diferenciales de variables separables.}
\author{Jorge Ruiz López}
\affil{Facultad de Ingeniería UNAM}
\date{Marzo 2022}
\begin{document}
\maketitle
\section{Ecuaciones diferenciales de variables separables.}
\begin{description}
\item[Definición 1.1:]Una ecuación diferencial  de primer orden de la forma:\\
\begin{center}$\frac{dy}{dx} = g(x)h(y)$\end{center}
Se dice que es \cite{zill}\textbf{\textit{separable}} o que tiene variables separables.\\\\
Observe que al dividir entre la función $h(y)$, podemos escribir una ecuación separable $\frac{dy}{dx} = g(x)h(y)$ como:
\begin{center}$p(y)\frac{dy}{dx}=g(x)$\end{center}\tab(2)\\\break
donde, por conveniencia $p(y)$ representa a $\frac{1}{h(y)}$.\\
Ahora si $y = f(x)$ representa una solución de la ecuación $(2)$, se tiene que 
$p(f(x))f'(x) = g(x)$, y por tanto
\begin{center}$\int (f(x))f'(x) = \int g(x) $\end{center}\tab(3)\\\break
Pero $dy = f'(x)dx$, por lo que la ecuación $(3)$ es la misma que
\begin{center}
    $\int p(y) dy = \int g(x) dx$\\o\\$H(y) = G(x)+c$
\end{center}\tab(4)\\\break
donde $H(y) y G(x)$ son antiderivadas de $p(y) = \frac{1}{h(y)}$ y $g(x)$, respectivamente.
\item[Método de solución:]La ecuación $(4)$ indica el procedimiento para resolver 
ecuaciones separables. Al integrar ambos lados de p(y) dy = g(x) dx, se obtiene una familia uniparamétrica de soluciones, que usualmente se expresa de manera implícita.\\
\textbf{Ejemplo:}\\Resuelva $(1 + x)dy - ydx = 0$
\begin{center}
    \textbf{Solución:}Dividiendo entre (1 + x)y, podemos escribir $\frac{dy}{y} = \frac{dx}{(1+x)}$, de donde tenemos que\\
    $\int \frac{dy}{y}$ $=$ $\int \frac{dx}{(1+x)}$\\
    $ln|y| = ln|1+x|+c_1$\\
    $y=e^{ln|1+x|+c_1}$ $=$ $e^{ln|1+x|}\cdot e^{c_1}$\\
    $=|1+x|e^{c_1}$\\
    $=\pm e^{c_1}(1+x)$\\
    Haciendo c igual a $e^{c_1}$ se obtiene \underline{ \textbf{$y=c\cdot (1+x)$}}\\
\end{center}
\end{description}
\printbibliography
\end{document}

