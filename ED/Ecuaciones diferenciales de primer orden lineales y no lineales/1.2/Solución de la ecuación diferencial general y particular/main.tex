\documentclass{article}
\usepackage[utf8]{inputenc}
\usepackage[english]{babel}
\usepackage{blindtext}
\usepackage{outlines}
\usepackage{authblk}
\usepackage{multicol}
\usepackage{geometry}
\usepackage{amssymb}
\usepackage{tabto}
\usepackage{biblatex} 
\addbibresource{ed.bib} %Import the bibliography file
 \geometry{
 letterpaper,
 left=25mm,
 top=25mm,
 bottom=25mm,
 right=30mm
 }
 
\title{ Solución de la ecuación diferencial: general y particular. Definción de soluciones}
\author{Jorge Ruiz López}
\affil{Facultad de Ingeniería UNAM}
\date{March 2022}

\begin{document}

\maketitle
\section{Solución de una ecuación diferencial:}
\begin{description}
\item[Definición 1.1:]\textbf{Solución}\cite{carmona} de una ecuación diferencial es una función que no contiene derivadas y que satisface a dicha ecuación; es decir, al sustituir la función y sus derivadas en la ecuación diferencial resulta una identidad.
\item[Definición 1.2:]\textbf{Solución general} de una ecuación diferencial es la función
que contiene una o más constantes arbitrarias (obtenidas de las sucesivas integraciones).\\
\textbf{Ejemplo:}
La función $x+y^2 = c$ es la \emph{solución general} de la ecuación diferencial:
\begin{center} $\frac{dy}{dx} = -\frac{1}{2y}$ \end{center}
Porque derivándola implícitamente tenemos: $1+2y\frac{dy}{dx} = 0$, o expresado en otra forma: $2yy' = -1$.
Sustituyendo y y y' obtenemos una identidad: 
\begin{center}${2\sqrt{c-x}} {(-\frac{1}{2\sqrt{c-x}})} = -1$ $\blacksquare  -1 = -1$;\\
donde $y=\sqrt{c-x}$\end{center}
\textbf{Ejemplo 2:} La función $y=3x^2+c_1x+c_2$ es \emph{solución general} de la ecuación diferencial $y''= 6$, porque \begin{center}$y'=6x +c_1$\\ y $y'' = 6$ $\blacksquare 6=6$\end{center}
\item[Definición 1.3:]\textbf{Solución particular} de una ecuación diferencial es la función cuyas constantes arbitrarias toman un valor específico.\\
\textbf{Ejemplo:} La función $y = e^{-x}+8$ es \emph{solución particular} de la ecuación diferencial 
$y'+e^x=0$, porque derivando la solución y sustituyéndola en la ecuación dada, obtenemos:
\begin{center}
$y'=-e^{-x}$\\$-e^{-x}+e^{-x}=0$ $\blacksquare 0 = 0$
\end{center}
\end{description}

\printbibliography

\end{document}
