\documentclass{article}
\usepackage[utf8]{inputenc}
\usepackage[spanish]{babel}
\usepackage{blindtext}
\usepackage{outlines}
\usepackage{authblk}
\usepackage{multicol}
\usepackage{geometry}
\usepackage{biblatex} 
\addbibresource{ed.bib} %Import the bibliography file
 \geometry{
 letterpaper,
 left=25mm,
 top=25mm,
 bottom=25mm,
 right=30mm
 }


\title{Definición de ecuación diferencial. Definición de orden de una ecuación diferencial.}
\author{Jorge Ruiz López}
\affil{Facultad de Ingeniería UNAM}
\date{Marzo 2022}

\begin{document}
\maketitle

\section{Definición de una ecuación diferencial}
\begin{description}
    \item[Definición 1.1]Una \emph{ecuación diferencial} \cite{carmona} es aquella ecuación que contiene derivadas o diferenciales.
     \begin{center}\textbf{Ejemplo:}\\
    Si tenemos: \quad $\frac{d^2x}{d^2y} = x$\\
    la llamamos ecuación diferencial de \textbf{segundo orden.}\\
    Integrando: \quad  $\frac{dx}{dy} = \frac{x^2}{2}+c_1$\\
    Si volvemos a integrar: $y=\frac{x^3}{6}+c_1x+c_2$\\
    obtenemos una \textbf{función-solución} que podemos comprobar al instante:\\
    derivando: $\frac{dy}{dx} = \frac{x^3}{2} +c_1$\\
    derivando de nuevo con respecto a x: $\frac{d^2x}{d^2y} = x$
    \end{center}
    \item[Definición 1.2]\textbf{Orden} de una ecuación diferencial es el de la derivada más alta contenida en ella.
    \item[Definición 1.3]\textbf{Grado} de una ecuación diferencial es la potencia a la que está elevada la derivada más alta, siempre y cuando la ecuación diferencial esté dada en forma polinomial. 
\end{description}

\section{Clasificación de las ecuaciones diferenciales}
\begin{description}
\item[\textbf{Tipo:}]
    \item[Ordinarias:]La ecuación diferencial contiene derivadas de una o más variables dependientes con respecto a una sola variable independiente.
     \item[Parciales:]La ecuación diferencial contiene derivadas parciales de una o más variables dependieiites con respecto a dos o más variables independientes. 
\end{description}

\begin{description}
    \item[\textbf{Orden:}]
    \item[Primer orden:]\begin{center}$f(x,y,y') = 0$\end{center}
    \item[Segundo orden:]\begin{center}$f(x,y,y',y'') = 0$\end{center}
    \item[Orden n:]\begin{center}$f(x,y,y',...,y^n) = 0$\end{center}
\end{description}
    
\begin{description}
    \item[\textbf{Grado:}]
    \item[Lineales]:\\a) La variable dependiente \emph{y} y todas sus derivadas son de 1er grado.\\
    b)Cada coeficiente de \emph{y} y sus derivadas depende solamente de la variable independiente x (puede ser constante).
\end{description}

\section{Ejemplo de ecuaciones diferenciales:}
\begin{center}
\begin{tabular}{|c|c|c|c|c|}
\hline
Ecuación & Tipo & Orden & Grado & Lineal \\
\hline
$\frac{dx}{dy} = 2e^{-x} $ & Ordinaria & 1 & 1& Si\\
\hline
$yy''+ x^2y = x$ & Ordinaria & 2 & 1& No\\
\hline
$\frac{\partial y}{\partial t} + \frac{\partial^2 y}{\partial^2 t} = c$ & Parcial & 2 & 1& Si\\
\hline
$y'+ y = \frac{x}{y} $ & Ordinaria & 1 & 1& No\\
\hline
$sen(y')+ y = 0 $ & Ordinaria & 1 & 1& No\\
\hline
\end{tabular}
\end{center}

\printbibliography


\end{document}
